\documentclass[15pt]{sprawozdanie}
\usepackage{soul}

\class{Praca dyplomowa inżynierska}
\title{Opracowanie wirtualnego środowiska do symulacji dynamiki lotu bezzałogowych statków powietrznych}
\author{\textbf{inż. Wojciech Gajda} 304494\\\vspace{20pt}\textbf{Igor Faliszewski} 313223}
\instructor{dr inż. Paweł Kotowski}
\deadline{\today}

\usepackage{graphicx}
\usepackage{amssymb}
\usepackage{fancyvrb}
\usepackage{amsmath}
\usepackage{polski}
\usepackage[utf8]{inputenc}
\usepackage{multicol}
\usepackage{subfig}
\usepackage{comment}
\usepackage{xcolor}

\begin{document}
\maketitle

\section*{Streszczenie}
W pracy opisano realizację systemu przeznaczonego do symulacji dynamiki lotu bezzałogowych statków powietrznych. System pozwala na prowadzenie symulacji lotu w czasie rzeczywistem, który dodatkowo jest prezentowany jest w postaci trójwymiarowej wizualizacji. W trakcie wykonywania lotu logowane są dane i mogą zostać wykorzystane do analizy lotu. Opracowany został uniwersalny model dynamiki pozwalający na swobodną konfigurację parametrów statku. Obejmuje on modyfikację właściwości mechanicznych, aerodynamicznych oraz konfigurację zespołów napędowych i~wpływu czynników zewnętrznych. Symulacja dynamiki rozszerzona została o system sterowania. System został zaprojektowany w sposóbu ułatwiający zmianę parameterów statków i symulacji, tworzenie nowych konfiguracji statków oraz tworzenie i strojenie systemów sterowania. W pracy zaprezentowane zostało działanie systemu dla przykładowych modeli latających: stałopłatowca, wielowirnikowca i rakiety.

\section*{Słowa kluczowe}

symulacja, grafika komputerowa 3D, bezzałogowy statek powietrzny, model dynamiki ruchu

\newpage

\section*{Abstract}

\section*{Keywords}

\newpage
\tableofcontents

\newpage



\color{red}

\section{Wstęp}

\subsection{O projekcie}

Symulacje komputerowe dynamiki ruchu stanowią użyteczne narzędzie w pracach inżynierskich. Pozwalają na analizę poprawności działania układu mechanicznego przed jego wyprodukowaniem. W szczególności w zagadnieniu jakim jest projektowanie bezzałogowych statków powietrznych, zastosowanie symulacji pozwala zminimalizować koszty wytworzenia poprawnie działającego systemu.


\subsection{Cel projektu}

Celem niniejszej pracy jest opracowanie wirtualnego środowiska do symulacji dynamiki lotu bezzałogowych statków powietrznych. System implementuje podstawowy model dynamiki statków powietrznych wyposażonych w silniki rotorowe, silniki odrzutowe, powierzchnie nośne i powierzchnie sterowe. Pozwala na przeprowadzenie lotu symulowanym obiektem ktorego parametery określane są przez konfiguracje. System dzieli sie na serwer i aplikacje kliencką. Róznorodność modułów pozwala na realizacje róznych scenariusz (strzał, upuszczenie ładunku, kolizje, wpływ warunków srodowiskowych etc.)

\subsection{Przegląd istniejących rozwiązań}

Historia symulatorów lotu sięga lat 30 XX wieku. Pierwotnie zastosowanie symulatorów sprowadzało się do szkolenia pilotów cywilnych i wojskowych. W znanej obecnie formie kompletne symulatory lotu stanowią rozbudowane systemy integrujące wysokiej klasy oprogramowanie z peryferiami mającymi wierne odwzorowanie kokpitu kierowanej maszyny. Symulatory wykorzystywane do treningu pilotów podlegają rygorystycznym regulacją prawnym i na ogół ich zadaniem jest odwzorowanie jednej konkretnej maszyny. Równolegle uproszczone wersje symulatorów zaczeły zyskiwać popularność w zastosowaniu cywilnym, jako element rozrywki. W szczególności gry komputerowe związane z lotem bardzo często poświęcały zgodność z model rzeczywistym na rzecz lepszych odczuć użytkownika.

Na rynku dostępnych jest wiele środowisk symulacyjnych o różnym stopniu szczegółowości. Pełne systemy lotu stanowią produkt komercyjny projektowany na indywidualne zamówienie. Do najpopularniejszych dostępnych systemów sprzedawanych jako zamkniete oprogramowanie należą m. in.:

\begin{itemize}
\item Microsoft Flight Simulator -  
\item VBS (Arma) - środowisko symulacyjne do wizualizacji pola walki
\item Warthunder - darmowa gra komputerowa wprowadzająca znaczną ilość historycznych i współczesnych modeli samolotów, których parametry zostały oparte na dostępnych i odtajnych danych.
\item RealFlight - modelarski symulator lotu
\end{itemize}

Istnieją również rozwiązania typu open-source, realizujące jedynie poszczególne zadania:

\begin{itemize}
\item JSBsim - rozbudowany silnik dynamiki lotu działający w czasie rzeczywistym  
\item Ardupilot, INAV, Betaflight - kompletne systemy sterowania dla modeli zdalnie sterowanych
\end{itemize}






\section{Specyfikacja}



\subsection{Architektura}

System został podzielony na moduły które robią blabla bla...

\begin{itemize}
\item UAV\_physic\_engine 
\item UAV\_controller 
\item UAV\_visualization 
\item UAV\_drop\_physic 
\item UAV\_aggregator 
\item UAV\_server 

\end{itemize}

\newpage

\section{Dobór technologii}

\subsection{UAV\_physic\_engine}

C++ - używamy C++ ze względu na wydajność i elastyczność

Eigen - blibkioteka do algebry liniowej i macierzowej

ZeroMQ - kolejka wiadomości

RapidXML

Cxxopts

\subsection{UAV\_visualization}

Java

LWJGL

Music

\subsection{UAV\_aggregator}

Rust

Nalgebra

....


\subsection{UAV\_server}

Docker

ubuntu:latest



\begin{comment}

Symulacje komputerowe dynamiki ruchu  W szczególności w zagadnieniu jakim jest projektowanie systemów sterowania do bezzałogowych statków powietrznych, zastosowanie takich narzędzi pozwala zminimalizować koszty wytworzenia poprawnie działającego systemu i przyśpieszyć jego rozwój i testowanie.

Celem niniejszej pracy jest opracowanie wirtualnego środowiska do symulacji dynamiki lotu bezzałogowych statków powietrznych. System implementuje podstawowy model dynamiki statków powietrznych wyposażonych w silniki rotorowe, silniki odrzutowe, powierzchnie nośne i powierzchnie sterowe. Pozwala na przeprowadzenie lotu symulowanym obiektem ktorego parametery określane są przez konfiguracje. System dzieli sie na serwer i aplikacje kliencką. Róznorodność modułów pozwala na realizacje róznych scenariusz (strzał, upuszczenie ładunku, kolizje, wpływ warunków srodowiskowych etc.)


\section{Model dynamiki statku powietrznego}

\begin{equation}
\begin{cases}
\dot{Y} =  T(Y) \cdot X + stabilizacja\\ 
A \cdot \dot{X} + \Omega (X) \cdot A \cdot X = F_g + F_a + F_d + F_o
\end{cases}
\end{equation}

gdzie:
\begin{itemize}
\item $Y$ - wektor pozycji i orientacji wyrażony w układzie globalnym
\item $X$ - wektor prędkości postępowej i kątowej w układzie związanym ze statkiem powietrznym
\item $A$ - macierz masowa
\item $\Omega$ - macierz gyroskopowa
\item $F_g$ - siła i moment pochodząca od siły grawitacji wyrażone w układzie związanym ze statkiem powietrznym
\item $F_a$ - siła i moment aerodynamiczny wyrażone w układzie związanym ze statkiem powietrznym
\item $F_d$ - siła i moment zespołów napędowych wyrażone w układzie związanym ze statkiem powietrznym
\item $F_o$ - siła i moment oddziaływań zewnętrznych napędowych wyrażone w układzie związanym ze statkiem powietrznym
\end{itemize}

\begin{equation}
Y = \begin{bmatrix}
x_{NED}\\
y_{NED}\\
z_{NED}\\
q_0\\
q_x\\
q_y\\
q_z
\end{bmatrix}
\end{equation}

\begin{equation}
X = \begin{bmatrix}
\dot{x}_b\\
\dot{y}_b\\
\dot{z}_b\\
P_b\\
Q_b\\
R_b
\end{bmatrix}
\end{equation}

\begin{equation}
F_g = R_{nb}(Y) \cdot  \begin{bmatrix}
0\\
0\\
g
\end{bmatrix}
\end{equation}

\begin{equation}
F_a = \frac{1}{2}\rho \cdot V_{tot}^2 S R_{wb} C_F \\
M_a = \frac{1}{2}\rho \cdot V_{tot}^2 S d R_{wb} C_F
\end{equation}

\end{comment}




\end{document}\grid
