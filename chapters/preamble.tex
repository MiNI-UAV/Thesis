\section{Wstęp}

\subsection{O projekcie}

Symulacje komputerowe dynamiki ruchu stanowią użyteczne narzędzie w pracach inżynierskich. Pozwalają na analizę poprawności działania układu mechanicznego przed jego wyprodukowaniem. W szczególności w zagadnieniu jakim jest projektowanie bezzałogowych statków powietrznych, zastosowanie symulacji pozwala zminimalizować koszty wytworzenia poprawnie działającego systemu.

\subsection{Przegląd istniejących rozwiązań}

Historia symulatorów lotu sięga lat 30. XX wieku. Pierwotnie zastosowanie symulatorów sprowadzało się do szkolenia pilotów cywilnych i wojskowych. W znanej obecnie formie kompletne symulatory lotu stanowią rozbudowane systemy integrujące wysokiej klasy oprogramowanie z peryferiami mającymi wierne odwzorowanie kokpitu kierowanej maszyny. Symulatory wykorzystywane do treningu pilotów podlegają rygorystycznym regulacją prawnym i na ogół ich zadaniem jest odwzorowanie jednej konkretnej maszyny. Równolegle uproszczone wersje symulatorów zaczęły zyskiwać popularność w zastosowaniu cywilnym, jako element rozrywki. W szczególności gry komputerowe związane z lotem bardzo często poświęcały zgodność z rzeczywistością na rzecz lepszych odczuć użytkownika.\\

Na rynku dostępnych jest wiele środowisk symulacyjnych o różnym stopniu szczegółowości. Pełne systemy lotu stanowią produkt komercyjny projektowany na indywidualne zamówienie. Do najpopularniejszych dostępnych systemów sprzedawanych jako zamknięte oprogramowanie należą m. in.:

\begin{itemize}
\item Microsoft Flight Simulator -  seria komputerowych symulatorów lotu pozwalająca na symulację pilotowania różnych statków powietrznych. Założeniem jest wierne odtworzenie zachowania statków powietrznych, warunków pogodowych, jak również samych maszyn,
\item VBS (Arma) - środowisko symulacyjne do wizualizacji pola walki,
\item Warthunder - darmowa gra komputerowa wprowadzająca znaczną ilość historycznych i współczesnych modeli samolotów, których parametry zostały oparte na dostępnych i odtajnionych danych,
\item RealFlight - modelarski symulator lotu.
\end{itemize}

Istnieją również rozwiązania typu open-source, realizujące jedynie poszczególne zadania:

\begin{itemize}
\item JSBsim - rozbudowany silnik dynamiki lotu działający w czasie rzeczywistym,
\item Ardupilot, INAV, Betaflight - kompletne systemy sterowania dla modeli zdalnie sterowanych.
\end{itemize}
