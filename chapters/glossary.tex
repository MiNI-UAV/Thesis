\section{Słownik pojęć}

\textbf{Statek powietrzny} -- urządzenie zdolne do unoszenia się (lotu) w atmosferze. Statek powietrzny jest zdolny w sposób aktywny wpływać na kierunek i prędkość lotu. W przeciwieństwie do formalnej definicji termin obejmuje również konstrukcję niewykorzystujące oddziaływania powietrza w locie (rakiety).\\

\textbf{Bezzałogowy statek powietrzny, BSP} -- statek powietrzny, który nie wymaga do lotu załogi obecnej na pokładzie oraz nie ma możliwości zabierania pasażerów, pilotowany zdalnie lub wykonujący lot autonomicznie.\\ 

\textbf{Pocisk} -- obiekt wystrzelony lub upuszczony ze statku powietrznego, nieposiadający własnego napędu. Porusza się na wskutek oddziaływania pola grawitacji i wpływu powietrza. Nie posiada wyróżnionej orientacji.\\

\textbf{Ładunek} -- pocisk, na ogół upuszczany, który pozostaje związany z statkiem powietrznym na sprężysto-tłumiącej linie. \\

\textbf{Stan obiektu} -- Opis położenia, orientacji i prędkości obiektu (pocisku lub statku powietrznego). Stan może zostać rozszerzony o dodatkowe informacje, takie jak: prędkości obrotowe poszczególnych silników, położenie powierzchni sterowych itd. \\

\textbf{Silnik fizyczny, silnik dynamiki} -- program komputerowy, którego zadaniem jest obliczenie położenia, orientacji i prędkości (kinematyki) statków powietrznych w zależności od sił działających na obiekt (dynamiki). \\

\textbf{Regulator} -- układ odpowiedzialny za generowanie rozkazów sterujących w oparciu o aktualny i zadany stan obiektu.\\

\textbf{Silnik graficzny} -- program komputerowy, którego zadaniem jest wizualizacja stanu obiektów i otoczenia.\\

\textbf{Agregator} --  program komputerowy, którego zadaniem jest zarządzaniem stanem aplikacji, obsługa przyłączających się aplikacji klienckich i zarządzanie procesami.\\

\textbf{Zasoby wizualizacji, assety} -- Modele i grafiki niezbędne do pracy aplikacji klienckiej.

\textbf{Fragment} -- Dana wymagana do wygenerowania pojedynczego piksela na ekranie. Zawiera m.in. zrasteryzowaną pozycję, głębokość oraz zinterpolowane atrybuty wierzchołków tworzących prymityw, które fragment jest częścią.

\textbf{Shader} -- Krótki program komputerowy, który w grafice trójwymiarowej pozwala na procesorze graficznym równolegle przetwarzać wierzchołki oraz fragmenty w celu uzyskania danych potrzebnych do wyświetlenia obrazu na ekranie.

\textbf{Bufor ramki, Frame buffer} -- Część pamięci RAM przeznaczona do przechowywania informacji o pojedynczej ramce obrazu. W buforze tym przechowywane są informacje o wartości każdego piksela tworzącego ramkę. 

\textbf{Teksel} -- Najmniejszy, dyskretny punkt tekstury. Tekstura jest macierzą tekseli, podobnie jak ekran jest macierzą pikseli.

