\chapter{Podział zadań w projekcie}

Realizacja projektu została podzielona między jego autorów. Do zadań poszczególnych autorów pracy należało:\\

\begin{itemize}
	\item Igor Faliszewski:
	\begin{itemize}[noitemsep,nolistsep]
		\item realizacja trójwymiarowej wizualizacji symulacji,
		\item implementacja graficznego interfejsu użytkownika,
		\item obsługa interakcji z użytkownikiem,
		\item wyposażenie aplikacji w zawartość,
		\item udźwiękowienie aplikacji.
	\end{itemize}
	\item Wojciech Gajda:
	\begin{itemize}[noitemsep,nolistsep]
		\item przygotowanie silnika dynamiki lotu BSP,
		\item przygotowanie silnika dynamiki obiektów niesterowalnych,
		\item przygotowanie systemu sterowania BSP,
		\item przygotowanie agregatora procesów i serwera symulacji,
		\item przygotowanie kontenera Docker z symulacją,
		\item przygotowanie generatora map.
	\end{itemize}
\end{itemize}
\ \\

Podczas przygotowywania tekstu pracy dyplomowej podział zadań odpowiadał zagadnieniom wdrażanym przez poszczególnych autorów. Elementy stałe pracy takie jak streszczenie, wstęp, specyfikacja i podsumowanie były wielokrotnie edytowane przez obu autorów. Wprowadzenie teoretyczne (rozdział \ref{theory}) zostało podzielone na dwie części, ze względu na swoją tematykę. Podrozdziały \ref{dynamika_lot} i \ref{sterowanie} przygotował Wojciech Gajda, natomiast podrozdziały \ref{grafika}, \ref{chapter_gui} i \ref{audio} został przygotowany przez Igora Faliszewskiego. W rozdziale Testy oprogramowania (rozdział \ref{testy}) podrozdziały dotyczące testów wizualizacji zostały przygotowane przez Igora Faliszewskiego, a pozostałe rozdziały przygotował Wojciech Gajda. Przygotowanie rozdziału \ref{wzdrozenie} Wdrożenie oprogramowania  zostało podzielone między autorów. Rozdziały dotyczące serwera oraz plików konfiguracyjnych serwera i BSP przygotował Wojciech Gajda, natomiast rozdziały dotyczące wizualizacji (klienta) oraz jej plików konfiguracyjnych przygotował Igor Faliszewski.

\newpage
Kod źródłowy projektu był rozwijany w wielu repozytoriach.
Tabela (\ref{rep_rep}) prezentuje osoby odpowiedzialne za poszczególne repozytoria.

\renewcommand{\arraystretch}{1.1}
\begin{table}[!h]
	\centering
	\begin{tabular}{|m{0.4\textwidth}|m{0.55\textwidth}|} 
		\hline
		\rowcolor{Gray}
		Repozytorium &  Osoba odpowiedzialna \\
		\multirow{1}{12em}{Igor Faliszewski} 
		& UAV\_visualization \\
		\hline
		\multirow{7}{12em}{{Wojciech Gajda}} 
		& UAV\_aggregator \\
		& UAV\_physics\_engine\\
		& UAV\_controller \\
		& UAV\_drop\_physic \\
		& UAV\_server \\
		& UAV\_map\_generator \\
		& UAV\_common \\
		\hline
	\end{tabular}
	\caption{Osoby odpowiedzialne za repozytoria}
	\label{rep_rep}
\end{table}