\section{Podsumowanie pracy}


W ramach pracy pomyślnie opracowano kompleksowy system umożliwiający symulację lotu bezzałogowych statków powietrznych. Wszystkie założone cele zostały osiągnięte, a zachowanie symulowanych statków nie wykazuje niezgodności z rzeczywistością.\\

Realizacja pracy pozwoliła poszerzyć umiejętności z zakresu projektowania i implementacji systemów komputerowych, ale także umożliwiła zanurzenie się w różnych aspektach lotnictwa oraz grafiki komputerowej. Zgromadzona wiedza obejmuje nie tylko techniczne aspekty tworzenia oprogramowania, lecz także praktyczne aspekty funkcjonowania bezzałogowych statków powietrznych.\\

Zbadane zostało wiele technik renderowania trójwymiarowej grafiki komputerowej dostępnych w specyfikacji OpenGL. Nie mniej jednak jest to jedynie wierzchołek góry lodowej jeżeli chodzi o możliwości tego interfejsu. Powyższą pracę można traktować jako porządne wprowadzenie do ogólnych technik renderowania grafiki trójwymiarowej. Będzie ona stanowić fundament wiedzy potrzebnej w przyszłych projektach wykorzystujących narzędzia pokrewne takie jak WebGL lub bardziej zaawansowane jak Vulkan. \\

Praca ta może być wykorzystana jako podstawa dalej rozwijanego systemu, zmodyfikowanego pod potrzeby konkretnego odbiorcy. Istnieje wiele zagadnień, które mogą w przyszłości zostać uwzględnione w rozwijanym systemie, a samego tematu z pewnością nie można uznać za zamknięty.