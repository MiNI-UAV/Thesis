\section{Wprowadzenie teoretyczne}

\epigraph{There is nothing so practical as a good theory.}{Lewin Kurt}
\epigraph{Nie ma osobnej ani teorii, ani praktyki inżynierskiej, jest tylko wspólna sztuka inżynierska.}{prof. Jan Oderfeld}

\subsection{Preliminaria?}
\subsubsection{Równania stanu}
\subsubsection{Metody numerycznego całkowania równań różniczkowych}
\subsection{Dynamika lotu BSP}
\subsubsection{Opis stanu BSP}
\subsubsection{Model matematyczny BSP}
\subsubsection{Siły i momenty działające na BSP}
\subsubsection{Model atmosfery ISA, wiatr}
\subsubsection{Kolizje}
\subsubsection{Odrzut}
\subsubsection{Przenoszenie ładunku}
\subsection{Sterowanie BSP}
\subsubsection{Regulacja i sterowanie}

\subsubsection{System sterowania BSP}
Regulacja i sterowanie, sterowanie w pętli zamkniętej z ujemnym sprzężeniem zwrotnym, model sterowania, drabina pidów
\subsubsection{Regulatory}
PID, LQR, inne.
\subsubsection{System nawigacji}
Czujniki, fuzja pomiarów, Filtr Kalmana

\subsection{Grafika komputerowa}
% Fit the following in the main chapter
%\subsubsection{Procesory graficzne}
%\subsubsection{OpenGL}
%\subsubsection{Potok renderowania}
%\subsubsection{Shadery}
%\subsubsection{Cieniowanie}
%\subsubsection{Model oświetlenia}
\subsubsection{Rysowanie oblamówki modelu za ścianą}
\subsubsection{Ślad po wystrzelonym pocisku}
\subsubsection{Rysowanie liny}
\subsubsection{Wyznaczenie parametrów panelu rotorów}
\subsubsection{Rysowanie radaru}
\subsubsection{Obsługa kontrolerów}

\begin{comment}

Symulacje komputerowe dynamiki ruchu  W szczególności w zagadnieniu jakim jest projektowanie systemów sterowania do bezzałogowych statków powietrznych, zastosowanie takich narzędzi pozwala zminimalizować koszty wytworzenia poprawnie działającego systemu i przyśpieszyć jego rozwój i testowanie.

Celem niniejszej pracy jest opracowanie wirtualnego środowiska do symulacji dynamiki lotu bezzałogowych statków powietrznych. System implementuje podstawowy model dynamiki statków powietrznych wyposażonych w silniki rotorowe, silniki odrzutowe, powierzchnie nośne i powierzchnie sterowe. Pozwala na przeprowadzenie lotu symulowanym obiektem którego parametry określane są przez konfiguracje. System dzieli się na serwer i aplikacje kliencką. Różnorodność modułów pozwala na realizacje rożnych scenariusz (strzał, upuszczenie ładunku, kolizje, wpływ warunków środowiskowych etc.)


\section{Model dynamiki statku powietrznego}

\begin{equation}
\begin{cases}
\dot{Y} =  T(Y) \cdot X + stabilizacja\\ 
A \cdot \dot{X} + \Omega (X) \cdot A \cdot X = F_g + F_a + F_d + F_o
\end{cases}
\end{equation}

gdzie:
\begin{itemize}
\item $Y$ - wektor pozycji i orientacji wyrażony w układzie globalnym
\item $X$ - wektor prędkości postępowej i kątowej w układzie związanym ze statkiem powietrznym
\item $A$ - macierz masowa
\item $\Omega$ - macierz gyroskopowa
\item $F_g$ - siła i moment pochodząca od siły grawitacji wyrażone w układzie związanym ze statkiem powietrznym
\item $F_a$ - siła i moment aerodynamiczny wyrażone w układzie związanym ze statkiem powietrznym
\item $F_d$ - siła i moment zespołów napędowych wyrażone w układzie związanym ze statkiem powietrznym
\item $F_o$ - siła i moment oddziaływań zewnętrznych napędowych wyrażone w układzie związanym ze statkiem powietrznym
\end{itemize}

\begin{equation}
Y = \begin{bmatrix}
x_{NED}\\
y_{NED}\\
z_{NED}\\
q_0\\
q_x\\
q_y\\
q_z
\end{bmatrix}
\end{equation}

\begin{equation}
X = \begin{bmatrix}
\dot{x}_b\\
\dot{y}_b\\
\dot{z}_b\\
P_b\\
Q_b\\
R_b
\end{bmatrix}
\end{equation}

\begin{equation}
F_g = R_{nb}(Y) \cdot  \begin{bmatrix}
0\\
0\\
g
\end{bmatrix}
\end{equation}

\begin{equation}
F_a = \frac{1}{2}\rho \cdot V_{tot}^2 S R_{wb} C_F \\
M_a = \frac{1}{2}\rho \cdot V_{tot}^2 S d R_{wb} C_F
\end{equation}

\end{comment}