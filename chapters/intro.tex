\section*{Streszczenie}
W pracy opisano realizację systemu przeznaczonego do symulacji dynamiki lotu bezzałogowych statków powietrznych. System pozwala na prowadzenie symulacji lotu w czasie rzeczywistym, który dodatkowo jest prezentowany w postaci trójwymiarowej wizualizacji. W trakcie wykonywania lotu logowane są dane i mogą zostać wykorzystane do analizy lotu. Opracowany został uniwersalny model dynamiki pozwalający na swobodną konfigurację parametrów statku. Obejmuje on modyfikację właściwości mechanicznych, aerodynamicznych oraz konfigurację zespołów napędowych i~wpływu czynników zewnętrznych. Symulacja dynamiki rozszerzona została o system sterowania. System został zaprojektowany w sposób ułatwiający zmianę parametrów statków i symulacji, tworzenie nowych konfiguracji statków oraz konfigurowanie i strojenie systemów sterowania. Przykładowych modelami, które mogą zostać zasymulowane są: stałopłatowiec, wielowirnikowiec i rakiety.

\section*{Słowa kluczowe}

symulacja, grafika komputerowa 3D, bezzałogowy statek powietrzny, model dynamiki ruchu

\newpage

\section*{Abstract}

The following paper describes the implementation of a system designed to simulate the flight dynamics of unmanned aerial vehicles. The system allows real-time flight simulation as well as three-dimensional visualization. During the flight we log the simulation data, allowing for flight analysis. A universal dynamics model has been developed to allow highly customizable aircraft parameters. Those include its mechanical and aerodynamic properties as well as the configuration of propulsion units and the influence of external factors. The dynamics simulation was extended to include the control system. The system has been designed in a way to make it easy to change ship and simulation parameters, create new configurations, and create and tune control systems. Examples of models that can be simulated include fixed-wind aircraft, multicopters and rockets.

\section*{Keywords}

simulation, 3D computer graphics, unmanned aerial vehicle, motion dynamics model

\newpage
\tableofcontents