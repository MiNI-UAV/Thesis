\section*{Streszczenie}
W pracy opisano realizację systemu przeznaczonego do symulacji dynamiki lotu bezzałogowych statków powietrznych. System pozwala na prowadzenie symulacji lotu w czasie rzeczywistym, który dodatkowo jest prezentowany jest w postaci trójwymiarowej wizualizacji. W trakcie wykonywania lotu logowane są dane i mogą zostać wykorzystane do analizy lotu. Opracowany został uniwersalny model dynamiki pozwalający na swobodną konfigurację parametrów statku. Obejmuje on modyfikację właściwości mechanicznych, aerodynamicznych oraz konfigurację zespołów napędowych i~wpływu czynników zewnętrznych. Symulacja dynamiki rozszerzona została o system sterowania. System został zaprojektowany w sposobu ułatwiający zmianę parametrów statków i symulacji, tworzenie nowych konfiguracji statków oraz tworzenie i strojenie systemów sterowania. Przykładowych modelami, które mogą zostać zasymulowane są: stałopłatowiec, wielowirnikowiec i rakiety. 

\section*{Słowa kluczowe}

symulacja, grafika komputerowa 3D, bezzałogowy statek powietrzny, model dynamiki ruchu

\newpage

\section*{Abstract}

\section*{Keywords}

\newpage
\tableofcontents